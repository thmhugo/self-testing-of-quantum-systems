\section{Robustness against white noise}


The aim of this section is to study robustness against the addition of a white noise. 

Consider a Werner state 

\begin{equation}
    \rho = \beta \ket{\Phi}\bra{\Phi} + (1-\beta) \mathbbm{1}
\end{equation}

where $\ket{\Phi}$ is a maximally entangled state and $\mathbbm{1}$ a fully randomized behaviour. 


\begin{procedure}
\begin{enumerate}
    \item Start with $\beta = 1$ and a given precision $\delta$
    \item While the result of the primal is non local for the state $\rho$ : $\beta = \beta - \delta $
\end{enumerate}
\end{procedure}

Procedure 1 is applied by a dichotomic algorithm in order to have a good precision with the fewer iterations possible. 


\input{Plot/plot_noise}

\vspace{1cm}

For the CHSH game, we obtained $\beta^*_{CHSH} = 1/\sqrt{2} $ and for Mayer-Yao's correlations, $\beta^*_{MY}= \approx 0.82$. In each case, one can noticed that $\beta^* = 1 - \alpha^*$ where $\alpha^*$ was the optimal objective obtained in Section .. The study of the robustness was already given by the primal detecting non-local behaviour since it was written using a convex combination of the behaviour and a fully randomized behaviour which is actually a white noise. However, if one wants to study another type of noise, this study would be necessary. 

A conclusion we can draw is that the maximally entangled state is more robust for CHSH correlations than for Mayer-Yao's correlations. 


