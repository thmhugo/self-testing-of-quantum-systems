\section{Non-locality detection}\label{Non-locality-det}

\subsection{Linear Programming formulation}\label{primal-LP}

\begin{proposition}
\label{convex_sum}
A behaviour $\mathbf{p}$ is local if and only if  it can be written as a convex sum of deterministic behaviours, i.e 
\begin{equation}
    \mathbf{p} = \displaystyle \sum_\lambda \mu_\lambda \mathbf{d}_\lambda \ , \ \mu_\lambda \geq 0 \ , \ \displaystyle \sum_\lambda \mu_\lambda = 1
\end{equation}
\end{proposition}

\begin{proposition}
\label{convex_sum2}
The set $\mathcal L$ of local behaviours is convex, i.e 
\begin{equation}
\mathcal P_1, \mathcal P_2  \in \mathcal L 
\Rightarrow \forall \alpha \in [0, 1] : \alpha \mathcal
P_1 + (1-\alpha) \mathcal P_2 \in \mathcal L
\end{equation}
\end{proposition}

Hence, it is possible to detect a non-local behaviour using linear programming. \\ Let $\mathcal P$ be the behaviour for which one want to learn whether it is local. Let $\mathbbm{1}$ be the behaviour corresponding to the random outcome strategy. It is clear that $\mathbbm{1}$ is a local behaviour. It is possible to write the following linear program 
\begin{equation}
    \displaystyle{\min_{\alpha,\overrightarrow{\mu}}} \, \alpha  \\

\left\{
\begin{array}{rcl}
 (1-\alpha) \mathcal{P}+ \alpha \mathbbm{1} &\ = \ & \displaystyle{\sum_\lambda} \mu_\lambda \mathbf{d_\lambda} \\
\displaystyle{\sum_\lambda} \mu_\lambda &\ = \ & 1  \\
\alpha  &\ \leq \ & 1 \\
\forall \lambda,\ \mu_\lambda \geq 0\ ,\ \alpha \geq 0  
\end{array}
\right.
\label{eq:primal}

\end{equation}

If the optimal value is $\alpha^* = 0$, then  $\mathcal{P}$ is a local behaviour since it can be written as 
\begin{equation*}
    \mathcal{P} = \displaystyle \sum_\lambda \mu^*_\lambda \mathbf{d_\lambda}
\end{equation*}
where $\mathbf{\mu^*}$ is the coefficients find at the optimum. \\
On the other hand, if  $\alpha^* > 0$, $ \mathcal{P} $ is non-local.\\

The variable $\alpha$ is linked to the visibility $V = 1 - \alpha$. 

\subsection{CHSH correlations}\label{CHSH_primal}

The behaviour $\mathcal{P}$ obtained by solving the system of equations induced by the quantum correlations (\textcolor{red}{put ref to introduction}) is 
\begin{equation*}
    \begin{tabular}{|c||*{4}{c|}}\hline
  $\mathcal{P}$   & \multicolumn{4}{ c| }{(x,y)} \\
  \hline
$(a,b)$
&\makebox[3em]{(0,0)}&\makebox[3em]{(0,1)}&\makebox[3em]{(1,0)}&\makebox[3em]{(1,1)}\\\hline\hline

(-1, -1) &$\cos^2(\pi/8)/2$&$\sin^2(\pi/8)/2$&$\sin^2(\pi/8)/2$&$\cos^2(\pi/8)/2$\\
\hline
(-1, 1) &$\cos^2(\pi/8)/2$&$\sin^2(\pi/8)/2$&$\sin^2(\pi/8)/2$&$\cos^2(\pi/8)/2$
 \\ 
 \hline
(1, -1)& $\cos^2(\pi/8)/2$&$\sin^2(\pi/8)/2$&$\sin^2(\pi/8)/2$&$\cos^2(\pi/8)/2$
 \\\hline
(1, 1) & $\sin^2(\pi/8)/2$&$\cos^2(\pi/8)/2$&$\cos^2(\pi/8)/2$&$\sin^2(\pi/8)/2$\\ 
 \hline
    \end{tabular}
\end{equation*}


The linear program \ref{eq:primal} gives an optimal 
\begin{equation}
    \alpha^*_{CHSH} = 1 - 1/\sqrt{2}
\end{equation}, meaning that the visibility is $V^*_{CHSH} = 1/\sqrt{2}$.


\subsection{Mayer-Yao's correlations}


The behaviour $\mathcal{P}$ obtained by solving the system of equations induced by the quantum correlations (\textcolor{red}{put ref to introduction}) is 

\begin{equation*}
    \begin{tabular}{|c||*{4}{c|}}\hline
  $\mathcal{P}$   & \multicolumn{4}{ c| }{$(a,b)$} \\
  \hline
$(x,y)$
&\makebox[3em]{(-1,-1)}&\makebox[3em]{(-1,1)}&\makebox[3em]{(1,-1)}&\makebox[3em]{(1,1)}\\\hline\hline

$(0,0)$ & $1/2$&$0$&$0$&$1/2$\\
\hline
$(0,1)$ &  $1/4$&$1/4$ &$1/4$&$1/4$ \\
\hline
$(0,2)$ & $\cos^2(\pi/8)/2$&$\sin^2(\pi/8)/2$&$\sin^2(\pi/8)/2$&$\cos^2(\pi/8)/2$ \\
\hline
$(1,0)$ & $1/4$&$1/4$&$1/4$&$1/4$ \\ 
\hline
$(1,1)$ & $1/2$&$0$&$0$&$1/2$ \\
\hline
$(1,2)$ &$\cos^2(\pi/8)/2$&$\sin^2(\pi/8)/2$&$\sin^2(\pi/8)/2$&$\cos^2(\pi/8)/2$ \\
\hline
$(2,0)$ & $\cos^2(\pi/8)/2$&$\sin^2(\pi/8)/2$&$\sin^2(\pi/8)/2$&$\cos^2(\pi/8)/2$ \\
\hline
$(2,1)$ & $\cos^2(\pi/8)/2$&$\sin^2(\pi/8)/2$&$\sin^2(\pi/8)/2$&$\cos^2(\pi/8)/2$ \\
\hline
$(2,2)$ & $1/2$&$0$&$0$&$1/2$ \\
 \hline
    \end{tabular}
    \end{equation*}

The linear program \ref{eq:primal} gives an optimal 
\begin{equation}
    \alpha^*_{MY} = 0.1715 
\end{equation}, meaning that the visibility is $V^*_{CHSH} = 0.8284$.


























